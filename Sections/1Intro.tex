This report presents an in-depth exploration of an optional project within the "Wireless and Mobile Propagation" course, taught by Professor Michele D’Amico at Politecnico di Milano.
The primary aim of this project is to cultivate proficiency in using "Radio Mobile," an advanced, free software application designed to simulate and predict radio network coverage areas.
The project encompasses several critical tasks.
Initially, users must configure various radio parameters and integrate ground elevation data into the software.

This data is vital for accurately simulating radio wave propagation.
With the appropriate settings and parameters, "Radio Mobile" can produce detailed coverage plots for individual base stations and combinations of multiple stations, which are essential for understanding the geographic reach and performance of radio networks.
The key input parameters necessary for predicting and generating a coverage map
in Radio Mobile include:
\begin{itemize}
    \item Transmitter location
    \item Transmitter power output
\item Frequency
\item Antenna type
\item Antenna pattern
\item Antenna gain
\item Transmission line losses
\item Receiver location
\item Receiver antenna type
\item Terrain and elevation data for the area
\end{itemize}

Radio coverage levels are displayed using various units, such as S-units, $\mu$V, dBm,
and $\mu$V/m.
For the purposes of this project, the universal dBm unit will be exclusively
used to represent radio coverage levels.
Additionally, "Radio Mobile" offers several tools that enhance its functionality.
One notable tool is the "radio link" feature, which provides comprehensive information on the link status between two stations, including critical metrics such as path loss, received power, and Fresnel zones.
These metrics are crucial for assessing the quality and reliability of radio links.
By setting these parameters, Radio Mobile delivers a detailed and accurate representation of radio coverage, enabling users to effectively assess and optimise the performance of their radio networks.
